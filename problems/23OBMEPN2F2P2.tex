desc:  $3$ parcelas de $3$ algarismos iguais
source:  OBMEP F2 2023/2
tags:  [divisibilidade, casework]

---

Carlinhos fez todas as adições possíveis com três parcelas diferentes, em que cada parcela é um número de três algarismos iguais, sempre colocando as parcelas em ordem crescente. Por exemplo, 222 + 555 + 888 e 444 + 777 + 888 foram adições feitas por Carlinhos. Ele não fez a adição 222 + 888 + 555, pois as parcelas não estão em ordem crescente, nem a adição 444 + 444 + 777, pois nela existem parcelas iguais.
a) Escreva uma adição que Carlinhos fez em que o resultado é 1332.
b) Escreva todas as adições que Carlinhos fez em que o resultado é 2109.
c) Explique por que 2109 é o único resultado das adições que Carlinhos fez em que o algarismo das dezenas é diferente do algarismo das centenas.

---

URL: https://drive.google.com/file/d/1mz2C09Ikpz5TZ9cjkzNlM_h8mHy-F90Z/view
